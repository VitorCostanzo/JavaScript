// Aritméticos
    //   + -> soma
    //   - -> subtração
    //   * -> multiplicação
    //   / -> divisão
    //   % -> resto da divisão inteira
    //   ** -> potenciação


// Mudança de conteudo
    // = -> atribui conteudo
    // a++ <=> (a = a + 1)
    // a-- <=> (a = a - 1)


// Comparação
    // ==  -> igual
    // !=  -> diferente
    // === -> estritamente igual
    // !== -> não estritamente igual
    // <   -> menor que
    // <=  -> menor ou igual a 
    // >   -> maior que
    // >=  -> maior ou igual a


// Lógicos
    // && -> e 
    // || -> ou 
    // !  -> não


// Bitwise (a nivel de bit)
    // &   -> operador e 
    // ||  -> operador ou
    // ^   -> operador ou exclusivo
    // ~   -> operador não
    // >>  -> rotação de bits para a direita
    // <<  -> rotação de bits para a esquerda
    // >>> -> rotação de bits para a direita sem levar em consideração o sinal


// Especiais
    // ternário    ->  condição ? resultadoSeVerdadeiro : resultadoSeFalso.
    // virgula     ->  a vírgula efetua operador da esquerda para a direita. O último elemento é retornado. Ex.: z=(x=1, y=2); x passa a valer 1, e y e z passam a valer 2.
    // delete      ->  limpa a referência, apaga a variável da memória.
    // in          ->  (propriedade in objeto) Retorna true caso a propriedade esteja contida no objeto.
    // instanceof  ->  (objeto instanceof TipoDoObjeto) Retorna true caso o objeto seja de determinado tipo.
    // typeof      ->  retorna uma string contendo o tipo do objeto.
    // new         ->  cria nova instância.
    // this        ->  representa a intancia do objeto corrente.
    // void        ->  void(expressão) Resolve a expressão e ignora valor retornado

referencia -> https://developer.mozilla.org/pt-BR/docs/Web/JavaScript/Guide/Expressions_and_operators